% Conclusions.tex
\chapter{Conclusions}

This project has successfully met its objective of developing a cost-effective and reliable analogue sun sensor system suitable for attitude determination in Low Earth Orbit (LEO) nanosatellite missions. Through the integration of hardware prototyping and software simulation, a robust methodology was established to evaluate the viability of photodiode-based sun sensing solutions.

A functional photodiode array was constructed, accompanied by a custom-designed signal conditioning circuit incorporating a transimpedance amplifier and post-amplification low-pass filtering. The system output was digitised using a Arduino-based data acquisition system, with post-processing conducted via Python to apply digital filtering and interpret results.

The software model, implemented in Python, and built on fundamental geometric principles, modelled ray-plane interactions using geometric approximation.  It features flexible configuration of sensor and aperture topologies, and supported visual and quantitative analysis of illumination results across a range of incident angles and positions. Conducting comparative analysis, between experimental (RED testbench) and simulated (Software model) data, validated the successful implementation of underlying principles and assumptions, by proving a strong correlation.

While the Renewable Energy Demonstrator (RED) testbench introduced some mechanical limitations, such as inconsistent servo performance and control noise, these were effectively mitigated through calibration, manual adjustments, and signal filtering techniques.

Ultimately, this project delivers a validated hybrid sun sensing system — combining practical prototyping with a software model.
% Your introduction content here
  