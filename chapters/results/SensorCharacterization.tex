% SensorCharacterization.tex
\section{Sensor Characterization}
For the SensorCharacterization.tex file, you'd want to focus on the fundamental properties and performance of your photodiodes themselves, distinct from the other subsections. Here are some key elements that would belong specifically under SensorCharacterization:

Basic Photodiode Electrical Characteristics:

Dark current measurements
Junction capacitance
I-V characteristics in different lighting conditions
Spectral response profiles (sensitivity vs. wavelength)


Individual Sensor Benchmarking:

Performance comparison between the 4 photodiodes (matching/differences)
Responsivity measurements (A/W)
Quantum efficiency calculations
Detection threshold levels


Response Linearity:

Measurements showing linear range of the photodiodes
Saturation point characterization
Recovery time from saturation


Temperature Dependency:

Performance drift with temperature
Baseline shift measurements
Temperature compensation data


Aging/Stability Tests:

Long-term drift measurements
Repeatability of measurements over time



This section should focus on the inherent properties of the photodiodes themselves - essentially providing the baseline characterization data that underpins all the other analysis. The other sections then build on this foundation by examining how these sensors perform when integrated into the complete system with amplification, angular positioning, enclosure effects, etc.

% Include a flowchart
\begin{figure}[H]
    \centering
    \scalebox{0.5}{ % Scale to 80% of original size
        % try generating flowcharts as svg in Claude 
% and edit with inkscape instead of this.
% but claude did generate this one so might 
% be useful too but you can't easily make
% small repairs in inkscape


% CNN Transfer Learning Flowchart - Compact Multi-Column Layout
% \begin{figure}[htbp]

\centering
\resizebox{\textwidth}{!}{ % Scale to fit width while maintaining aspect ratio
\begin{tikzpicture}[node distance=0.8cm and 1.5cm, auto]
    % Define a smaller block style
    \tikzset{
      block/.style = {rectangle, draw, fill=blue!20, 
                      text width=7em, text centered, rounded corners, minimum height=1.8em, font=\small},
    }
    
    % Brazilian model training - Column 1
    \node [block] (brazildata) {Download Brazilian coins dataset};
    \node [block, below=of brazildata] (extract) {Extract dataset};
    \node [block, below=of extract] (setup) {Setup directories};
    \node [block, below=of setup] (define) {Define train/val dirs};
    \node [block, below=of define] (create) {Create CNN architecture};
    \node [block, below=of create] (compile) {Compile the CNN};
    \node [block, below=of compile] (train) {Train model};
    \node [block, below=of train] (trained) {Model trained (5 classes)};
    
    % Transfer learning - Column 2 (Middle)
    \node [block, right=2.5cm of brazildata] (freeze) {Freeze all layers};
    \node [block, below=of freeze] (replace) {Replace final layers};
    \node [block, below=of replace] (add) {Add regularization and dropout};
    \node [block, below=of add] (output) {New output layer (8 classes)};
    \node [block, below=of output] (finaltrain) {Train and fine-tune};
    \node [block, below=of finaltrain] (inference) {Perform inference on new coins};
    
    % UK data preparation - Column 3 (Right)
    \node [block, right=2.5cm of freeze] (ukdata) {Download UK coins dataset};
    \node [block, below=of ukdata] (ukextract) {Extract UK dataset};
    \node [block, below=of ukextract] (uksetup) {Setup UK directories};
    \node [block, below=of uksetup] (ukgen) {Create data generators (80/20 split)};
    
    % Connect all nodes with arrows
    \path [line] (brazildata) -- (extract);
    \path [line] (extract) -- (setup);
    \path [line] (setup) -- (define);
    \path [line] (define) -- (create);
    \path [line] (create) -- (compile);
    \path [line] (compile) -- (train);
    \path [line] (train) -- (trained);
    
    \path [line] (ukdata) -- (ukextract);
    \path [line] (ukextract) -- (uksetup);
    \path [line] (uksetup) -- (ukgen);
    
    % Connect the columns
    \path [line] (trained) -- node[midway, above] {Transfer} (freeze);
    \path [line] (ukgen) |- (finaltrain);
    
    % Connect middle column
    \path [line] (freeze) -- (replace);
    \path [line] (replace) -- (add);
    \path [line] (add) -- (output);
    \path [line] (output) -- (finaltrain);
    \path [line] (finaltrain) -- (inference);
    
    % Group boxes to show different stages with smaller padding
    \begin{pgfonlayer}{background}
        \node[group={[yshift=0.3cm]above:Brazilian Model Training}, fit={(brazildata) (extract) (setup) (define) (create) (compile) (train) (trained)}, inner sep=0.2cm] {};
        \node[group={[yshift=0.3cm]above:UK Data Preparation}, fit={(ukdata) (ukextract) (uksetup) (ukgen)}, inner sep=0.2cm] {};
        \node[group={[yshift=0.3cm]above:Transfer Learning}, fit={(freeze) (replace) (add) (output) (finaltrain) (inference)}, inner sep=0.2cm] {};
    \end{pgfonlayer}
\end{tikzpicture}
}
% \caption{CNN Transfer Learning Flowchart: Brazilian to UK Coins}
% \label{fig:cnn-flowchart}
% \end{figure}
    }
    \caption{System Design Overview Flowchart}
    \label{fig:decriptiveLabel2} % descriptive to call in text with \ref{fig:decriptiveLabel}
\end{figure}

\subsection{Functional Requirements}
% Your content here

\subsection{Design Approach}
% Your content here

\subsection{System Architecture}
As shown in Figure~\ref{fig:decriptiveLabel2} the system architecture consists of various components.

\begin{lstlisting}[style=cstyle, caption=System Architecture Code Example, label=lst:SystemArchitecture2]
# Your code here
\end{lstlisting}

\begin{figure}%[htbp] %h-ere t-op b-ottom p-page (separte) -good to allow all htbp to give the compiler more options
    \centering
    \includegraphics[width=0.6\textwidth]{figures/results/system_architecture.jpg}
    \caption{System Architecture Diagram}
    \label{fig:system-architecture24}
\end{figure}
