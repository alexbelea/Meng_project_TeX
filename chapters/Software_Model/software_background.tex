\section{Introduction}
A Python model was constructed to provide a simulation of the movement and intersection of rays from a movable source to evaluate sensor performance and compare these results with practical experiments.
The model allows for a number of configurable parameters:
\begin{itemize}
    \item The trajectory of the light source 3D space, which moves in configurable discrete increments.
    \item The placement of any number of sensors and apertures, including their dimensions.
    \item The form of the output data, including as a static, or animated graphic.
\end{itemize}
Affording flexibility for the model to simulate any sensor topology under a variety of conditions.

% State the theory
\section{Theory and Concept}
% Ray projection
The system is modelled in 3D space, consisting of planes and lines. 
Each line is defined by vectors representing position and normal direction, $ \vec{A} = (a,b,c)$ and $ \vec{u} = (\alpha, \beta, \gamma)$.
The planes are defined by the vectors $\vec{P} = (l,m,n) $ and $ \vec{n} = (\lambda, \mu, \nu)$, respectively.

Ray projection, from a source plane to a sensor plane, is modelled using the parametric equation of a 3D line (\ref{eq:ray_projection}). This allows each ray to be described in terms of a parameter $t$, which enables the calculation of the intersection points between the light rays and the sensor plane. 
% Intersection equation
\begin{equation}
\frac{x - a}{\alpha} = \frac{y - b}{\beta} = \frac{z - c}{\gamma} (=t)
\label{eq:ray_projection}
\end{equation}

Where the intersection coordinates $(x,y,z)$ occur within a target area, a hit occurs, representing illumination.

For any given combination of source plane, and sensor plane, the $t$ parameter is calculated using the Line-Plane Intersection equation

\begin{equation}
    t = \frac{\vec{n} \cdot \vec{P} - \vec{n} \cdot \vec{A}}{\vec{n} \cdot \vec{u}}
    \label{Line-Plane Intersection}
\end{equation}