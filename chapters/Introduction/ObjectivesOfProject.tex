\section{Objectives of the Project}
To investigate the design of a sun sensing system for nanosatellites, used in orientation determination, through detection of its relative position to the sun using analogue sensors located on the satellite's body. Our goal is to create a system which balances cost-effectiveness and simplicity.
To achieve this, we will create a software model of the analogue sensor(s) to simulate the system's ability to track the sun from various angles in orbit. After which, we aim to build a physical prototype and use a movable light source to simulate the sun's movement, allowing comparison between the real sensor's performance against our simulations.
Although the physical prototype will be built using non-space-grade materials, one of the objectives is to look at and analyse materials required for building a space-grade PCB and sensor. For this step, the Mechanical side of the team will perform Printed circuit board (PCB) and aperture device finite analysis using ANSYS to determine resilience to environmental factors such as stress and thermal simulation.
Throughout the project, we will address challenges like interference from other light sources (such as the moon), reflections from the Earth's surface, and how factors like radiation and temperature changes in space might impact the sensor's accuracy. The application of signal processing will be explored to provide usable data, filter out noise, and improve the system's accuracy.
This approach aims to develop a cost-effective and reliable, in-house sun sensing solution specifically for nanosatellites operating in Low Earth Orbit.
\begin{itemize}
    \item \textbf{Conduct literature review:}
    \begin{itemize}
    \item Analyse existing research on sun sensing technologies, with a focus on PSD-based analogue sensors and their applications in nanosatellites.
    \item Identify current challenges, best practices, and advancements in attitude determination in Low Earth Orbit. Use these insights to guide the design and optimisation of the proposed sun sensing system.
    \end{itemize}
    \item \textbf{Develop software model:}
\begin{itemize}
    \item Simulate the performance of the PSD-based analogue sun sensor in tracking the sun's position from various angles in Low Earth Orbit.
\end{itemize}

\item \textbf{Design and fabrication of physical prototype:}
\begin{itemize}
    \item Integrate analogue sun sensor components, test and validate its performance under controlled conditions.
\end{itemize}

\item \textbf{Compare simulated and experimental results:}
\begin{itemize}
    \item Establish evaluation methodology between simulated and experimental test results to ensure that topology evaluation is applicable.
\end{itemize}

\item \textbf{Optimise sensor topology:}
\begin{itemize}
    \item Research and evaluate various configurations of analogue sun sensing systems to maximise sun detection accuracy and minimise blind spots.
\end{itemize}

\item \textbf{Investigate environmental factors:}
\begin{itemize}
    \item Evaluate the impact of relevant LEO specific environmental factors on the sensor's accuracy and reliability.
    \item Evaluate the material requirements of the PCB and aperture device.
\end{itemize}

\item \textbf{Implement signal processing algorithms:}
\begin{itemize}
    \item Investigate the filtering of noise to enhance the signal-to-noise ratio and otherwise ensure the acquisition of usable data for accurate sun position determination.
    \item Implement data handling which optimises scanning rates and efficiently processes the analogue signal data for real-time attitude determination.
\end{itemize}

\item \textbf{Document results and overall cost-effectiveness:}
\begin{itemize}
    \item Develop criteria for final evaluation of sun sensing systems, on which to base the final presentation of project findings
\end{itemize}