% Conclusions.tex
\chapter{Conclusions}

This project has successfully met most of the objectives of developing a cost-effective and reliable analogue sun sensor system showing the ability to detect light position which could be useful for future research in developing simpler, and more cost effective Sun Sensors for low cost \acf{LEO} attitude determination of nanosatellite missions. Through the integration of hardware prototyping and software simulation, a robust methodology was established to evaluate the viability of photodiode-based sun sensing solutions.

A photodiode array was designed and built together with the required current amplification circuitry. The output voltage was then recorded using a custom made \ac{DAQ} that was correctly able to collect the four signals sequentially for post processing and analysis and results interpretation.


The \acf{RED} testbench introduced some issues in the project, such as the servo motors struggling to match wanted angles, which created difficulties in recording angles consistently, requiering manual adjustments and removed the ability to get full hemisphere readings for a \ac{LUT} that would be used to interpolate new readings.

The software model, implemented in Python, facilitated accurate simulation of ray-plane interactions using geometric approximation. It enabled flexible configuration of sensor and aperture topologies, and supported visual and quantitative analysis of illumination results across a range of incident angles and positions. Comparative analysis between experimental and simulated data demonstrated strong correlation, validating the fidelity of the model and the underlying assumptions.

Ultimately, this project delivers a validated hybrid sun sensing system — combining practical prototyping with a software model.
