\chapter{Future Work}
%Mention: methods to avoid detecting sun reflections off the moon and earth (such as light intensity or light source width if possible).

%Unfortunately, there was no other easily available method of fabricating the apertures, such as having the apertures printed on glass with a fully opaque ink. Another method considered and attempted was to 3D print the apertures, but this would have resulted in the apertures having a thickness that would have changed the way the light enters, depending on the angle of the light.

\paragraph{Enhanced Aperture Design and Manufacturing}

The current prototype utilized manually placed apertures, resulting in alignment inconsistencies that affected measurement accuracy. 
Future iterations should explore precision manufacturing techniques such as photolithography or laser etching to create apertures with consistent placement and sharper edges. 
Additionally, testing alternative aperture geometries could optimize the sensor's field of view and accuracy across different incident angles.

\paragraph{Advanced Calibration Techniques}

Development of a comprehensive Look-Up Table (LUT) and interpolation algorithm would enable more precise attitude determination. 
Future work should include deploying the sensor on a high-precision platform to map responses across the entire field of view, generating a detailed calibration map.

\paragraph{Enviromnetla Qualification and Testing}

While thermal simulations suggested the viability of the design in space conditions, physical testing in thermal vacuum chambers would validate the sensor's performance under actual space-like conditions. 
Radiation testing would also be necessary to assess component degradation over an extended mission lifetime.

\paragraph{Discrimination capabilities}

Methods to avoid detecting sun reflections from the Earth, Moon, or spacecraft components should be investigated. 
Possibilities include implementing intensity thresholds, spectral filtering, or temporal signature analysis to distinguish direct solar illumination from reflected sources.