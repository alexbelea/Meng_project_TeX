% abstract.tex
\chapter*{Abstract}
This research project explores the design, implementation, and testing of a prototype of a cost-effective photodiode array-based analogue 2D sun sensor for attitude determination in Low Earth Orbit (LEO) nanosatellites, with a focus on the CubeSat variety. 
As the commercialisation of space continues to grow, there is a demand for low-cost, reliable attitude determination systems for small satellites that cannot accommodate the expensive digital camera systems used in larger commercial missions.
A comprehensive review of existing sun sensor technologies, CubeSat designs, mechanical considerations and signal analysis methods has been undertaken.

Based on this foundation, a prototype was developed utilising four photodiodes arranged in a T-shaped configuration with appropriate apertures, coupled with transimpedance and secondary amplification circuitry to process the photodiode signals. 
The prototype was housed in a custom-designed 3D-printed enclosure to ensure proper positioning of the photodiodes and protection of the electronic components.
A software model was created in parallel to simulate ray projection and intersection calculations, allowing for the prediction of sensor response under various light conditions. 
This model provided valuable insights for optimising the physical design and interpreting experimental results. 
Additionally, material analysis was performed to evaluate suitable materials for space deployment. Polyimide was identified as the optimal PCB material due to its balance of thermal stability, radiation resistance, and mechanical properties. 
Thermal analysis using ANSYS confirmed that the selected components would function reliably within the extreme temperature range of space environments from \SI{200}{\celsius} to \SI{-200}{\celsius}.
A Data Acquisition System (DAQ) based on an Arduino microcontroller was implemented to record and process the sensor data, incorporating digital filtering to eliminate noise and enhance signal quality. 
Testing was conducted using a Renewable Energy Demonstrator (RED) as a testbench to position a light source at precise angles.
The research demonstrates the feasibility of developing a low-cost sun-sensing solution for nanosatellites that balances simplicity with adequate performance for attitude determination in space applications. 
The findings contribute to the growing field of small satellite technology and offer potential pathways for future improvements in sun sensor design for space missions
\addcontentsline{toc}{chapter}{Abstract}