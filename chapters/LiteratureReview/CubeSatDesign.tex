\section{CubeSat Design}
Puig-Suari, Turner and Ahlgren published an IEEE paper in 2001 with the help of their students at California Polytechnic State University exploring a need for micro satellites for use by universities in an ever-expanding space programme. 
They provide as a solution a standard satellite form-factor that will bring down the cost of both manufacture and deployment of satellites by smaller entities: the CubeSat. 
The paper identifies a key component for the success of this form factor a need for a standard CubeSat deployer mechanism which can deploy several satellites safely and develop such a platform, called Poly Picosatellite Orbital Deployer or P-POD. 
They point out the need and provide microsatellite size and shape of the CubeSat form factor \cite{RefWorks:puig-suari2001development}.

Sai balaji et al. performed a study using MATLAB simulation of several attitude control algorithms to look at the ability to control a CubeSat of size 1U. 
They also simulated sensors such as sun sensors, magnetometer, and gyroscope. 
They concluded that it is possible to operate the satellite using a magnetorquer type actuator and an array of mathematical models and algorithms: it would take 2000 seconds for a 1U satellite to stabilize at 505km, 98° degree attitude in orbit with the methods utilized by them \cite{RefWorks:balaji2023studies}.

Incentivised by the rapidly increasing use of LEO, Lopez‑Calle and Franco perform a quantitative comparative study on the catastrophic failure of CubeSats and Nanosats from radiation exposure due to the harsh environment of space versus failure due to collisions in the increasingly busy Low Earth Orbit (LEO). 
The authors concluded that while sustained damage and damage protection from radiation exposure used to be and currently still is the most crucial factor in protecting LEO microsatellites, increasingly the risk of debris collisions is becoming more important and will become the most important in the following 50 to 70 years. 
The authors conclude that microsatellite designers need to move their focus more towards defence from debris impacts as these, even if not resulting in catastrophic failure of the satellite, they will impact the attitude of the satellite \cite{RefWorks:lopez-calle2023comparison}.