% State the theory
\subsection{Implementation}

The software model implements the theoretical foundation to provide a flexible framework for simulating ray projection and intersection calculations. The implementation follows a modular object-oriented approach with  components that can be configured to represent different experimental setups.

\subsubsection{Architecture Overview}

The architecture consists of several key components:
\begin{itemize}
\item Core geometric objects (Planes, Areas, Lines) that encapsulate the mathematical properties
\item Simulation engine for trajectory generation and intersection testing
\item Configuration system for experiment setup
\item Visualisation  pipeline for result analysis
\item User interface for interactive control
\end{itemize}

\subsubsection{Model Configuration}

A .JSON file is used to allow for easy setup of different experimental scenarios without modifying code:

\begin{itemize}
\item Detailed definition of planes, including position, orientation, and dimensions
\item Sensor and aperture areas with specific positions and sizes
\item Trajectory specifications for various movement types
\item Simulation parameters (number of rays, iterations)
\item Visualisation options - Static, or animated plot 
\item Debugging and performance settings
\end{itemize}

The \texttt{Config} class (Listing~\ref{lst}) loads and parses the configuration file, making parameters accessible throughout the application:
\begin{lstlisting}[style=pythonstyle, caption=Model configuration - Config Class, label=lst:pythonCodeApp, language=Python ]
    
    class Config:
        def init(self, file_path=None, data=None):
        if data is not None:
        self.load_from_dict(data)
        elif file_path:
        with open(file_path, "r") as f:
        data = json.load(f)
        self.load_from_dict(data)
        else:
        raise ValueError("Must provide either file_path or data")
    
    def load_from_dict(self, data):
        self.planes = data["planes"]
        self.sensor_areas = data["sensor_areas"]
        self.aperture_areas = data["aperture_areas"]
        self.arc_movement = data["arc_movement"]
        self.simulation = data["simulation"]
        self.intersection = data["intersection"]
        self.visualization = data["visualization"]
        self.debugging = data["debugging"]
        self.performance = data["performance"]
        self.output = data["output"]
    \end{lstlisting}